\documentclass[oupdraft]{bio}
%\usepackage[colorlinks=true, urlcolor=citecolor, linkcolor=citecolor, citecolor=citecolor]{hyperref}
% \documentclass[12pt,a4paper]{article}
\pdfoutput=1
\usepackage[utf8]{inputenc}
\usepackage{algorithm}
\usepackage{algpseudocode}
%\usepackage[]{graphicx}
%\usepackage{amsmath}
%\usepackage{amssymb}
%\usepackage[]{color}
%\usepackage{subcaption}
% \usepackage{natbib}
%\usepackage{xspace}
%\usepackage{calc}
% \usepackage[noae]{Sweave}
%\usepackage{algorithm}
%\usepackage{algpseudocode}
% \usepackage{enumitem}

% \usepackage{authblk}

% \usepackage[nolists,nomarkers]{endfloat}


%\usepackage{geometry}
%\geometry{a4paper, portrait, margin=1in}
%
% New commands
%
%\newcommand{\ie}{\emph{i.e.}\@\xspace}
%\newcommand{\eg}{\emph{e.g.}\@\xspace}

\newcommand{\nn}{\nonumber}

% \newcommand{\mcomment}[1]{\textcolor{red}{\textbf{#1}}}

\DeclareMathOperator{\E}{\mathbb{E}}
\DeclareMathOperator{\Var}{Var}
\DeclareMathOperator{\Cov}{Cov}
\newcommand{\argmin}{\operatornamewithlimits{arg\,min}}
\newcommand{\plim}{\operatornamewithlimits{plim}}

%\usepackage[amsthm,thmmarks]{ntheorem}
% \theoremstyle{plain}
% \newtheorem{theorem}{Theorem}
%\newtheorem{definition}{Definition}
%\newtheorem{proposition}{Proposition}
\newtheorem{corollary}{Corollary}[section]
%\newtheorem{lemma}{Lemma}
%\newtheorem{assumption}{Assumption}
%\newtheorem{example}{Example}
%\renewcommand\proofSymbol{\ensuremath{\blacksquare}}



%%% RSS specific items

%\title{Sequential rank agreement methods for comparison of ranked lists}
%\author{Claus Thorn Ekstrøm}
%\author{Thomas Alexander Gerds}
%\author{Andreas Kryger Jensen}
%\author{Kasper Brink-Jensen}
%\affil{Biostatistics, University of Copenhagen}
%\renewcommand{\topfraction}{.85}
%\renewcommand{\bottomfraction}{.7}
%\renewcommand{\textfraction}{.15}
%\renewcommand{\floatpagefraction}{.66}
%\renewcommand{\dbltopfraction}{.66}
%\renewcommand{\dblfloatpagefraction}{.66}
%\setcounter{topnumber}{9}
%\setcounter{bottomnumber}{9}
%\setcounter{totalnumber}{20}
%\setcounter{dbltopnumber}{9}


\renewcommand{\topfraction}{.85}
\renewcommand{\bottomfraction}{.7}
\renewcommand{\textfraction}{.15}
\renewcommand{\floatpagefraction}{.66}
\renewcommand{\dbltopfraction}{.66}
\renewcommand{\dblfloatpagefraction}{.66}
\setcounter{topnumber}{9}
\setcounter{bottomnumber}{9}
\setcounter{totalnumber}{20}
\setcounter{dbltopnumber}{9}


\begin{document}


\title{Supplementary material for Sequential rank agreement methods for comparison of ranked lists}

\author{CLAUS THORN EKSTRØM$^\ast$, THOMAS ALEXANDER GERDS, ANDREAS KRYGER JENSEN\\[4pt]
%
\textit{%
Biostatistics, Department of Public Health,
University of Copenhagen,
Øster Farimagsgade 5 B, P.O.B. 2099,
DK-1014 Copenhagen K, Denmark}
\\[2pt]
% E-mail address for correspondence
{ekstrom@sund.ku.dk}}



% Use square brackets for numbering affiliations


% Running headers of paper:
\markboth%
% First field is the short list of authors
{C. T. Ekstrøm and others}
% Second field is the short title of the paper
{Supplementary material for ``Sequential rank agreement for comparison of ranked lists''}

\maketitle


This document provides proofs of the theorem and corollary shown in the main manuscript as well as details about the algorithm for incomplete lists.

\appendix

% \newpage
\section{Proof of Theorem 3.1}
\label{sec:appA}

We start by defining the following cadlag function
\begin{align}
  N(p; d) = \sum_{s=1}^p 1\left(Q(R(X_s) \leq d) > \varepsilon\right), \quad p = 1,\ldots, P
\end{align}
which runs through the list elements in an arbitrary order and counts
how many list elements that have a positive probability $>\varepsilon$ to have
rank less than or equal to $d$ under the probability measure $Q$ for some prespecified constant $\varepsilon\in[0, 1)$.
Note that the cardinality of the set $S(d)$ in equation
(2.3) is equal to the counting process evaluated at the last element: $|S(d)| = N(P;d)$.
The empirical counterpart of the counting process is given by
\begin{align}
  \widehat{N}_L(p; d) &= \sum_{s=1}^p 1\left( \frac{1}{L} \sum_{l=1}^L 1(R_l(X_s) \leq d) > \varepsilon\right)\\
        &= \sum_{s=1}^p 1\left(\widehat{Q}_L(R(X_s) \leq d) > \varepsilon\right)\nn.
\end{align}
The joint law of the sequence
$\{\widehat{N}_L(p;d)\}_{1\leq p \leq P}$ is completely determined by
the finite set of the jump times of $N(p;d)$. Since
$\{1(R_l(X_s) \leq d); l=1,\dots, L\}$ consists of independent and
identically Bernoulli distributed variables with expectation
$Q(R(X_s) \leq d)$ and finite variance it follows from the law of large numbers that
$\widehat{Q}_L(R(X_p) \leq d) \overset{P}{\longrightarrow} Q(R(X_p)
\leq d)$ for every $p$ and $d$ as $L \rightarrow \infty$. 
Therefore, we have
\begin{align}
  \sup_{d \in 1,\ldots,P}\sup_{p \in 1,\ldots, P} \left|\widehat{N}_L(p; d) - N(p; d)\right| = o_P(1).\label{eq:countingConv}
\end{align}

The sequential rank agreement given in equation
(2.5) may therefore be rewritten as the following integral
with respect to $N$
\begin{align}
  \textrm{sra}(d) = \int_{1}^P \frac{A(X_p)}{N(P; d)} 1(N(P;d)>0)\mathrm dN(p; d).
\end{align}
The empirical sequential rank agreement is similarly given by
\begin{align}
  \widehat{\textrm{sra}}_L(d) &= \int_1^P \frac{\widehat{A}_L(X_p)}{\widehat{N}_L(P; d)} 1(N(P;d)>0) \mathrm d \widehat{N}_L(p; d) 
\end{align}
and it follows that
\begin{align}
  \sup_{d \in 1,\ldots,P}\left|\widehat{\textrm{sra}}_L(d) - \textrm{sra}(d)\right| \leq \sup_{d \in 1,\ldots,P}\left|U_L(d)\right| - \sup_{d \in 1,\ldots,P}\left|V_L(d)\right|
\end{align}
where
\begin{align}
  U_L(d) &= \int_1^P\frac{A(X_p)}{N(P;d)} 1(N(P;d)>0) \mathrm d\left(\widehat{N}_L(p; d) - N(p; d)\right)\\
  V_L(d) &=  \int_1^P \left(\frac{\widehat{A}_L(X_p)}{\widehat{N}_L(P; d)} - \frac{A(X_p)}{N(P;d)}\right) 1(N(P;d)>0) \mathrm d\widehat{N}_L(p; d)
\end{align}
The conclusion of the proof therefore follows if each of these two
terms are uniformly of order $o_P(1)$ in $d$.  For the first term we
have that
\begin{align}
  \left|U_L(d)\right| \leq \left(\sup_{p \in 1,\ldots,P}\left|\frac{A(X_p)}{N(P;d)} 1(N(P;d)>0) \right|\right)\left(\widehat{N}_L(p; d) - N(p; d)\right)\biggr\rvert_1^P
\end{align}
where the second factor is $o_P(1)$ uniformly in $d$ by equation
(\ref{eq:countingConv}). The first factor is $O(1)$ since $N(P;d)>0$
uniformly in $d$ and because $A(X_p) = O(P^2)$ uniformly in $p$ by the Cauchy-Schwarz inequality. It
thus follows that $\left|U_L(d)\right| = o_P(1)$ uniformly in $d$.

Similarly we derive an upper bound for $V_L(d)$ by
\begin{align}
 V_L(d) &\leq \sup_{p \in 1,\ldots,P}\left|\left(\frac{\widehat{A}_L(X_p)}{\widehat{N}_L(P; d)} - \frac{A(X_p)}{N(P;d)}\right) 1(N(P;d)>0) \right| \int_1^P \left|\mathrm d\widehat{N}_L(p; d)\right|\label{eq:secondTerm}
\end{align}
and it follows directly from the law of large numbers that
$\widehat{A_L}(X_p) = A(X_p) + o_P(1)$ uniformly in $p$ since
$R_i(X_p)$ in equation (2.2) for $i=1,\ldots,L$ are
independent and $Q$-identically distributed random variables with
finite second moment. Again, $\widehat{N}_L(P; d) = N(P;d) + o_P(1)$ uniformly in $d$ by equation
(\ref{eq:countingConv}) and $N(P;d)>0$ so by the continuous mapping theorem the first factor
in equation (\ref{eq:secondTerm}) is of order $o_P(1)$ uniformly in
$p$ and $d$. The result then follows by noting that the second factor
is bounded by $P$ uniformly in $d$. 




\section{Proof of Corollary 3.1}
\label{sec:appB}
Recall that $\mathcal{L}$ is the superlevel set of list positions
where the sequential rank agreement exceeds the threshold function
$q$. Let $A \triangle B = (A \setminus B) \cup (B \setminus A)$ be the
symmetric difference between sets $A$ and $B$. It is then sufficient
to show that $\widehat{\mathcal{L}}_L(\widehat{q}_L) \triangle \mathcal{L}(q) \overset{P}{\longrightarrow} \emptyset$
for $L \rightarrow \infty$ where $\emptyset$ denotes the empty set.
We have that
\begin{align}
  \plim_{L \rightarrow \infty}\left(\widehat{\mathcal{L}}_L(\widehat{q}_L) \triangle \mathcal{L}(q)\right) &=   \left(\plim_{L \rightarrow \infty}\widehat{\mathcal{L}}_L(\widehat{q}_L)\right) \triangle \mathcal{L}(q)\\
  &= \left\{d : \plim_{L\rightarrow \infty} \left(\widehat{\textrm{sra}}_L(d) - \widehat{q}_L(d)\right)\geq 0\right\} \triangle \mathcal{L}(q)\nn\\
  &= \left\{d : \textrm{sra}(d) - q(d) \geq 0\right\}\triangle \mathcal{L}(q)\nn\\
  &= \mathcal{L}(q)\triangle \mathcal{L}(q)\nn\\
  &= \emptyset\nn
\end{align}
as a consequence of Theorem 3.1, the assumption that $\left\|\widehat{q}_L - q\right\|_\infty = o_P(1)$ and the continuous mapping theorem. This completes the proof.


\section{Algorithm for sequential rank agreement for incomplete lists}
\label{sec:appA}



\begin{algorithm}
\caption{Sequential rank agreement algorithm for incomplete lists}
\label{sra-algorithm}
\begin{algorithmic}[1]
  \Procedure{Incomplete list rank agreement}{} \State Let $B$ be the number
  of permutations to use \For{each $b \in B$} \For{each list
    $l \in L$}
  \State \parbox[t]{\dimexpr\linewidth-\algorithmicindent-1.7cm}{Permute
    the unassigned ranks, $\{d_l+1, \ldots, P\}$, and assign them
    randomly to the items \emph{not} found in the list, i.e.,
    $\Lambda^\complement=X\setminus\Lambda_l$. Combine the
    result with $\Lambda_l$ to fill out the list.}
\EndFor
\State \parbox[t]{\dimexpr\linewidth-\algorithmicindent-1cm}{Let sra($b$) be the sequential rank agreement computed from the
filled out lists.}
\EndFor
\State Return element-wise averages across all $B$ permutations of sra($b$).
\EndProcedure
\end{algorithmic}
\end{algorithm}

\end{document}

